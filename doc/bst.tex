\documentclass{article}

\usepackage{cite}

\title{Some notes on binary search trees}
\date{\today}
\author{David M. Doolin}


\begin{document}

\maketitle

\abstract{A few notes on binary search trees and their implementations
in various programming languages.}



\section{Introduction}


The following topics will be discussed:

\begin{itemize}

\item Differences in implementations between programming languages.
\item How best to containerize, including pure tree node, tree wrapper with
node class, abstracting keys.
\item Theory and performance.

\end{itemize}


\section{Literature review}

Skiena~\cite[pp. 77, 370, 375, 589]{skiena} has a few notes.

Aho and Ullman~\cite[pp. 210]{rosen} define height and depth for nodes in a binary tree.

\begin{quote}
The height of a node n is the length of a longest path from n to
a leaf. The height of the tree is the height of the root. The depth or level of
a node n is the length of the path from the root to n.
\end{quote}

Cormen et al.~\cite{cormen:th:1990} remains a classic reference.

Rosen~\cite[pp. 757-760]{rosen} discusses three uses for binary search trees:

\begin{enumerate}
\item storing items from a list such that they can be easily found;
\item finding an object in a collection of similar objects;
\item efficiently encoding characters in a bit string.
\end{enumerate}

Sedgewick~\cite{sedgewick:r1990}.

O'Rourke~\cite{orourke:j1998}.

Manber~\cite[p. 87, Ex. 4.8]{manber:u1989} provides an interesting exercise:
show the AVL tree formed by inserting the ordered sequence $[1, 2, \ldots, 20]$.

\subsection{Collisions or duplicate values}

None the preceding references explicitly discuss collisions.

\begin{itemize}
\item The duplicate key can be added to the tree, either silently or with a warning.
\item The duplicate key is not added to the tree, either silently or with a warning.

\item In Ruby, does replacing a node in a tree leak memory, or will the deleted node
get garbage collected? How about Python?

\item How to delete in c/c++ such that memory isn't leaked? This should be easier,
call delete after the node is orphaned.
\end{itemize}

\section{Persistence}

Here are some ways to store the structure in text format:

\begin{enumerate}
\item Write to nested json. This requires writing out from the top
down. Should be able to do this by converting to hash, then
writing to json. Reading in is the reverse. How to convert a
hash to a binary tree? Maybe this is where to use combinator.

\item CSV: does this require top down, or can it be done by traversing?
Does it require a parent pointer?

\item Yaml: how are references handled?
\end{enumerate}


\section{Breadth-first traversal}

Breadth-first traveral can be used for searching, for determining whether a key exists,
or for writing out the tree in flat format with rows and pointers.

For persistence, we have the following:

\begin{itemize}
\item At each node we can store pointers to each child.
\item We can store the start and the stop of each row separately.
\item We can store an array of arrays, where each of the arrays is
      one level of the tree.
\end{itemize}

[
 [root],
 [ root.left, root.right],
 [...]
]

Can all the pointers be stored in one pass, then process in a second pass,
or can the nodes be processes as the breadth-first traverse proceeds?



\section{Containerizing}

\subsection{C++ templating}

\subsection{C struct inclusion}

\subsection{Ruby module inclusion}

It's not difficult to encode binary search tree capability into a Ruby module.
The module provides all the necessary instance binary search tree methods, and
raises errors when necessary overriding has not been implemented in the
including class.

The usual caveats about instance variables referenced in modules applies,
particularly that the including class must define the key for the tree.

\section{References}

\bibliography{references}{}
\bibliographystyle{plain}

\section{Summary}

\appendix

\section{Implementation}


\subsection{Testing}

Two ways of arranging tests include:
\begin{enumerate}
\item Group tests by method or function, ensuring a number of different
trees pass the method being tested.
\item Group tests by the tree, ensuring each operation on that particular tree
executes correctly.
\end{enumerate}

Traditionally, in Test-Driven Design (TDD), a method test is created with a simple example.
The method is then implemented as simply as possible to ensure the example passes.
More complex examples are created, and the method's implementation modified as necessary
to ensure tests pass.

Grouping tests by data structure examples puts a much larger burden on TDD,
but provides a different perspective. These tests may be more effective once an
initial implementation is created. Then, pathological data structures can be
created and tested in detail.

\end{document}
