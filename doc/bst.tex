\documentclass{article}

\usepackage{cite}

\title{Some notes on binary search trees}
\date{\today}
\author{David M. Doolin}


\begin{document}

\maketitle

\abstract{A few notes on binary search trees and their implementations
in various programming languages.}



\section{Introduction}


The following topics will be discussed:

\begin{itemize}

\item Differences in implementations between programming languages.
\item How best to containerize, including pure tree node, tree wrapper with
node class, abstracting keys.
\item Theory and performance.

\end{itemize}


\section{Literature review}

Skiena~\cite[pp. 77, 370, 375, 589]{skiena} has a few notes.


Rosen~\cite[pp. 757-760]{rosen} discusses three uses for binary search trees:

\begin{enumerate}
\item storing items from a list such that they can be easily found;
\item finding an object in a collection of similar objects;
\item efficiently encoding characters in a bit string.
\end{enumerate}

\subsection{Collisions}

None the preceding references explicitly discuss collisions.


\section{Persistence}

Here are some ways to store the structure in text format:

\begin{enumerate}
\item Write to nested json. This requires writing out from the top
down. Should be able to do this by converting to hash, then
writing to json. Reading in is the reverse. How to convert a
hash to a binary tree? Maybe this is where to use combinator.

\item CSV: does this require top down, or can it be done by traversing?
Does it require a parent pointer?

\item Yaml: how are references handled?
\end{enumerate}


\section{Containerizing}

\subsection{C++ templating}

\subsection{C struct inclusion}

\subsection{Ruby module inclusion}

Not sure if this will work, but it would be cool to have a Ruby module which
adds binary search tree capability to any object. The methods would have to be
class methods.

\section{References}

\bibliography{references}{}
\bibliographystyle{plain}

\section{Summary}

\appendix



\end{document}
